\documentclass[12pt]{scrartcl}

%For bibliography:
\usepackage{cite}
\usepackage{natbib}

%For lists and bullet points:
\usepackage{blindtext}
\usepackage{scrextend}

%\setcitestyle{square}

%\usepackage[hmarginratio=1:1,top=32mm,columnsep=20pt]{geometry} % Document margins

\title{Motivation Letter} % Article title
\author{Martin Bonev}
\date{\today}


\begin{document}

% Print the title
\maketitle

%The actual text:
I would like to apply for a PhD program in Musicology at the Department of Music Acoustics (IWK) at the University of Music and Performing Arts Vienna.

I have always had a passion for sound and music. I started playing the guitar when I was eight years old and this has had a profound effect on my life. I have been a part of several bands and ensembles and I have performed multiple times on stage.  I have learned that communication between the musicians is vital for every ensemble. The performers need to be able to easily anticipate each other's actions as well as telegraph their own in order to progress the music forward with precise temporal coordination. I believe that further research into this area, such as the Coordination and Collaborative Creativity in Music Ensembles (COCREATE) project, which I am part of, could lead to a breakthrough in the understanding of ensemble music performance. Furthermore, undertaking a PhD program as well would allow me to conduct a more thorough research, gain more relevant knowledge knowledge and extend my skill-set. Such a development as a professional and as a person is something I value immensely. 

A large part of the PhD program as well as the COCREATE project is the development of a computer accompaniment system (CAS). The development of such a system requires skills in three separate areas - music, mathematics and computer science.

I completed a BMus Music Technology degree in the University of Edinburgh, in which I gained essential knowledge of music theory and composition. I have taken courses such as Music Analysis and Listening and Musicianship where I have learned how to analyze and how to critically and intently listen to a piece of music. Furthermore, I learned how to play a number of instruments - guitar, bass and piano and have played in a few bands as a guitarist and a bass player. All this experience in music making and performance would allow me to construct and test the CAS from a musician's point of view.

During my BMus Music Technology degree, I greatly increased my knowledge in mathematics as well as in acoustics and digital audio. I learned to express the behaviour of acoustic systems, such as strings and drums skins, with differential equations and other mathematical methods. I also gained experience in digital signal processing, digital filter design and implementation of sound effects, including pitch/time shift and phaser. Additionally, I have composed musical pieces based on mathematical functions, which is also known as "Algorithmic Composition". In those pieces, the chord progression and the melodic line were primarily constructed from the outcome of either Markov chains or Lindenmayer systems (L-systems). Such a mathematical background would give me a great advantage in applying the methods of machine learning in the algorithm design for the computer accompaniment system.

Furthermore, I have completed an MSc degree in Computer Science at the University of Edinburgh, in which the focus was on Software Engineering and High Performance Computing. I gained knowledge of software design and testing and developed crucial skills in programming in a number of languages, including C/C++, Java, and MATLAB. I have also improved my abilities in code debugging and build automation, through the use of tools such as the GNU Debugger (GDB) and Make. Furthermore, I have experience in both multi-threaded programming with the OpenMP API and Java’s Thread Class/Runnable Interface and GPU programming with the CUDA and OpenCL APIs. The skills I developed during my Computer Science degree would allow me to design and build the CAS as a piece of software that is robust and easy to maintain.

I believe that a PhD program focused on developing a computer accompaniment system (CAS) to test and explore human-human and human-computer music performance interaction would be relevant for the Department of Music Acoustics (IWK) at the University of Music and Performing Arts Vienna. The results from this research could significantly change the perception of group music performance, while a functional CAS would be a helpful and convenient tool for musicians to practice with. Furthermore, the COCREATE project is undertaken in collaboration with the Intelligent Music Processing and Machine Learning Group at the Austrian Research Institute for Artificial Intelligence (OFAI), which would be immensely helpful for the development of the accompaniment software. Finally, a PhD program in Musicology would be an excellent opportunity for me to apply my knowledge in both software development and music performance and greatly improve and extend my skill-set.



%I believe I have the relevant skills and experience to take part in a PhD program at the Department of Music Acoustics (IWK) at the University of Music and Performing Arts Vienna. Participating in the COCREATE project, would be an excellent opportunity for me to apply my knowledge in software development, further improve my skill-set and help in the development of the musical accompaniment system that will be used to test human- computer musical collaboration. I believe that the results from this research could significantly change the perception of group music performance.

% I would welcome the opportunity of an interview to discuss my application.
%\bibliography{proposalbib}{}
%\bibliographystyle{plain}

\end{document}